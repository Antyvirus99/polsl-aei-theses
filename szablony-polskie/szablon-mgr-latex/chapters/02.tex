\chapter{[Analiza tematu]}


\begin{itemize}
\item analiza tematu
\item wprowadzenie do dziedziny (\ang{state of the art}) – sformułowanie problemu, 
\item poszerzone studia literaturowe, przegląd literatury tematu (należy wskazać źródła wszystkich informacji zawartych w pracy)
\item opis znanych rozwiązań, algorytmów, osadzenie pracy w kontekście
\item Tytuł rozdziału jest często zbliżony do tematu pracy. 
\item Rozdział jest wysycony cytowaniami do literatury \cite{bib:artykul,bib:ksiazka,bib:konferencja}. 
Cytowanie książki \cite{bib:ksiazka}, artykułu w czasopiśmie \cite{bib:artykul}, artykułu konferencyjnego \cite{bib:konferencja} lub strony internetowej \cite{bib:internet}.
\end{itemize}
