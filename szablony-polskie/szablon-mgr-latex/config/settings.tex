
%%%%%%%%%%%%%%%%%%%%%%%%%%%%%%%%%%%%%%%%%%%%%%%
%                                             %
% PROSZĘ NIE MODYFIKOWAĆ PONIŻSZYCH USTAWIEŃ! %
%                                             %
%%%%%%%%%%%%%%%%%%%%%%%%%%%%%%%%%%%%%%%%%%%%%%%



\documentclass[a4paper,twoside,12pt]{book}
\usepackage[utf8]{inputenc}                                      
\usepackage[T1]{fontenc}  
\usepackage{amsmath,amsfonts,amssymb,amsthm}
\usepackage[british,polish]{babel} 
\usepackage{indentfirst}



\usepackage{ifxetex}

\ifxetex
	\usepackage{fontspec}
	\defaultfontfeatures{Mapping=tex—text} % to support TeX conventions like ``——-''
	\usepackage{xunicode} % Unicode support for LaTeX character names (accents, European chars, etc)
	\usepackage{xltxtra} % Extra customizations for XeLaTeX
\else
	\usepackage{lmodern}
\fi



\usepackage[margin=2.5cm]{geometry}
\usepackage{graphicx} 
\usepackage{hyperref}
\usepackage{booktabs}
\usepackage{tikz}
\usepackage{pgfplots}
\usepackage{mathtools}
\usepackage{geometry}
\usepackage[page]{appendix} % toc,
\renewcommand{\appendixtocname}{Dodatki}
\renewcommand{\appendixpagename}{Dodatki}
\renewcommand{\appendixname}{Dodatek}

\usepackage{csquotes}
\usepackage[natbib=true,backend=bibtex]{biblatex}
\bibliography{biblio/biblio}

\usepackage{ifmtarg}   % empty commands  

\usepackage{setspace}
\onehalfspacing


\frenchspacing

%%%%%%%%%%%%%%%%%%%%%%%%%%%%%%%%%%%%%%%%%%%%%%%%%%%%%%%%%%%%%%%%%%%%%
% listingi i fragmentu kodu źródłowego 
% pakiet: listings lub minted
% % % % % % % % % % % % % % % % % % % % % % % % % % % % % % % % % % % 

% biblioteka listings
\usepackage{listings}
\lstset{%
language=C++,%
commentstyle=\textit,%
identifierstyle=\textsf,%
keywordstyle=\sffamily\bfseries, %\texttt, %
%captionpos=b,%
tabsize=3,%
frame=lines,%
numbers=left,%
numberstyle=\tiny,%
numbersep=5pt,%
breaklines=true,%
%morekeywords={descriptor_gaussian,descriptor,partition,fcm_possibilistic,dataset,my_exception,exception,std,vector},%
escapeinside={@*}{*@},%
}

% % % % % % % % % % % % % % % % % % % % % % % % % % % % % % % % % % % 
% pakiet minted
%\usepackage{minted}

% pakiet wymaga specjalnego kompilowania:
% pdflatex -shell-escape praca

%%%%%%%%%%%%%%%%%%%%%%%%%%%%%%%%%%%%%%%%%%%%%%%%%%%%%%%%%%%%%%%%%%%%%


%%%% TODO LIST GENERATOR %%%%%%%%%

\usepackage{color}
\definecolor{brickred}      {cmyk}{0   , 0.89, 0.94, 0.28}

\makeatletter \newcommand \kslistofremarks{\section*{Uwagi} \@starttoc{rks}}
  \newcommand\l@uwagas[2]
    {\par\noindent \textbf{#2:} %\parbox{10cm}
{#1}\par} \makeatother


\newcommand{\ksremark}[1]{%
{%\marginpar{\textdbend}
{\color{brickred}{[#1]}}}%
\addcontentsline{rks}{uwagas}{\protect{#1}}%
}

\newcommand{\comma}{\ksremark{przecinek}}
\newcommand{\nocomma}{\ksremark{bez przecinka}}
\newcommand{\styl}{\ksremark{styl}}
\newcommand{\ortografia}{\ksremark{ortografia}}
\newcommand{\fleksja}{\ksremark{fleksja}}
\newcommand{\pauza}{\ksremark{pauza `--', nie dywiz `-'}}
\newcommand{\kolokwializm}{\ksremark{kolokwializm}}
\newcommand{\cudzyslowy}{\ksremark{,,polskie cudzysłowy''}}

%%%%%%%%%%%%%% END OF TODO LIST GENERATOR %%%%%%%%%%%

%%%%%%%%%%%% ZYWA PAGINA %%%%%%%%%%%%%%%
% brak kapitalizacji zywej paginy
\usepackage{fancyhdr}
\pagestyle{fancy}
\fancyhf{}
\fancyhead[LO]{\nouppercase{\it\rightmark}}
\fancyhead[RE]{\nouppercase{\it\leftmark}}
\fancyhead[LE,RO]{\it\thepage}


\fancypagestyle{tylkoNumeryStron}{%
   \fancyhf{} 
   \fancyhead[LE,RO]{\it\thepage}
}

\fancypagestyle{bezNumeracji}{%
   \fancyhf{} 
   \fancyhead[LE,RO]{}
}

\fancypagestyle{NumeryStronNazwyRozdzialow}{%
   \fancyhf{} 
   \fancyhead[LE]{\nouppercase{\FirstName\ \Surname}}
   \fancyhead[RO]{\nouppercase{\leftmark}} 
   \fancyfoot[CE, CO]{\thepage}
}


%%%%%%%%%%%%% OBCE WTRETY  
\newcommand{\obcy}[1]{\emph{#1}}
\newcommand{\ang}[1]{{\selectlanguage{british}\obcy{#1}}}
%%%%%%%%%%%%%%%%%%%%%%%%%%%%%

% polskie oznaczenia funkcji matematycznych
\renewcommand{\tan}{\operatorname {tg}}
\renewcommand{\log}{\operatorname {lg}}

% jeszcze jakies drobiazgi

\newcounter{stronyPozaNumeracja}

%%%%%%%%%%%%%%%%%%%%%%%%%%% 
\usepackage{xstring}
\usepackage{ifthen}
\newcommand{\printOpiekun}[1]{%		

    \StrLen{\Consultant}[\mystringlen]
    \ifthenelse{\mystringlen > 0}%
    {%
       {\large{\bfseries OPIEKUN, PROMOTOR POMOCNICZY}\par}
       
       {\large{\bfseries \Consultant}\par}
    }%
    {}
} 
%
%%%%%%%%%%%%%%%%%%%%%%%%%%%%%%%%%%%%%%%%%%%%%%
 
% Proszę nie modyfikować poniższych definicji!
\author{\FirstName\ \Surname}
\newcommand{\Author}{\FirstName\ \MakeUppercase{\Surname}}
\newcommand{\Type}{PRACA MAGISTERSKA}
\newcommand{\Faculty}{Wydział Automatyki, Elektroniki i Informatyki} 
\newcommand{\Polsl}{Politechnika Śląska}
\newcommand{\Logo}{graf/politechnika_sl_logo_bw_pion_pl.pdf}
\newcommand{\LeftId}{Nr albumu}
\newcommand{\LeftProgram}{Kierunek}
\newcommand{\LeftSpecialisation}{Specjalność}
\newcommand{\LeftSUPERVISOR}{PROWADZĄCY PRACĘ}
\newcommand{\LeftDEPARTMENT}{KATEDRA}
%%%%%%%%%%%%%%%%%%%%%%%%%%%%%%%%%%%%%%%%%%%%%%

