% !TeX spellcheck = en_GB
%
%%%%%%%%%%%%%%%%%%%%%%%%%%%%%%%%%%%%%%%%%%
%                                        %
%     Master thesis LaTeX template       % 
%  compliant with the SZJK regulations   %
%                                        %
%%%%%%%%%%%%%%%%%%%%%%%%%%%%%%%%%%%%%%%%%%
%                                        %
%  (c) Krzysztof Simiński, 2018-2022     %
%                                        %
%%%%%%%%%%%%%%%%%%%%%%%%%%%%%%%%%%%%%%%%%%
%                                        %
% The latest version of the templates is %
% available at                           %
% github.com/ksiminski/polsl-aei-theses  %
%                                        %
%%%%%%%%%%%%%%%%%%%%%%%%%%%%%%%%%%%%%%%%%%
%
%
% This LaTeX project formats the final thesis 
% with compliance to the SZJK regulations. 
% Please to not change formatting (fonts, margins,
% bolds, italics, etc).
%
% You can compile the project in several ways.
%
% 1. pdfLaTeX compilation
% 
% If you use the minted package for code snippets
% compile the project like this:
%
% pdflatex -shell-escape main
% bibtex                 main
% pdflatex -shell-escape main
% pdflatex -shell-escape main
%
% If you do not use the minted package, just comment 
% the package import \usepackage{minted} and its use in the thesis.
% Then run the compilation:
%
% pdflatex main
% bibtex   main
% pdflatex main
% pdflatex main
%
%
% 2. XeLaTeX compilation
%
% Compilation with the XeLaTeX engine inserts Calibri font
% in the title page. Of course the font has to be installed.
% As above you can use the minted package or not. 
% Compilation with the minted package:
%
% xelatex -shell-escape main
% bibtex                main
% xelatex -shell-escape main
% xelatex -shell-escape main
%
% Without the minted package the compilation is simpler:
%
% xelatex main
% bibtex  main
% xelatex main
% xelatex main
%
%
% manuals for code snippets packages:
% https://ctan.org/pkg/minted
% https://ctan.org/pkg/listings
%
%%%%%%%%%%%%%%%%%%%%%%%%%%%%%%%%%%%%%%%%%%%%%%%%%%%%%%%%%%%%%%
% If you have any questions, remarks, just send me an email: %
%            krzysztof.siminski(at)polsl.pl                  %
%%%%%%%%%%%%%%%%%%%%%%%%%%%%%%%%%%%%%%%%%%%%%%%%%%%%%%%%%%%%%%




%%%%%%%%%%%%%%%%%%%%%%%%%%%%%%%%%%%%%%%%%%%%%%%
%                                             %
% SETTINGS                                    %
% DO NOT MODIFY                               %
%                                             %
%%%%%%%%%%%%%%%%%%%%%%%%%%%%%%%%%%%%%%%%%%%%%%%


\documentclass[a4paper,twoside,12pt]{book}
\usepackage[utf8]{inputenc}                                      
\usepackage[T1]{fontenc}  
\usepackage{amsmath,amsfonts,amssymb,amsthm}
\usepackage[polish,british]{babel} 
\usepackage{indentfirst}



\usepackage{ifxetex}

\ifxetex
	\usepackage{fontspec}
	\defaultfontfeatures{Mapping=tex—text} % to support TeX conventions like ``——-''
	\usepackage{xunicode} % Unicode support for LaTeX character names (accents, European chars, etc)
	\usepackage{xltxtra} % Extra customizations for XeLaTeX
\else
	\usepackage{lmodern}
\fi



\usepackage[margin=2.5cm]{geometry}
\usepackage{graphicx} 
\usepackage{hyperref}
\usepackage{booktabs}
\usepackage{tikz}
\usepackage{pgfplots}
\usepackage{mathtools}
\usepackage{geometry}
\usepackage[page]{appendix} % toc,




\usepackage{csquotes}
\usepackage[style=numeric,backend=bibtex]{biblatex}
\bibliography{biblio}       % biblatex

\usepackage{ifmtarg}   % empty commands  

\usepackage{setspace}
\onehalfspacing


\frenchspacing

%%%%%%%%%%%%%%%%%%%%%%%%%%%%%%%%%%%%%%%%%%%%%%%%%%%%%%%%%%%%%%%%%%%%%
% listings
% packages: listings or minted
% % % % % % % % % % % % % % % % % % % % % % % % % % % % % % % % % % % 

% package listings
\usepackage{listings}
\lstset{%
language=C++,%
commentstyle=\textit,%
identifierstyle=\textsf,%
keywordstyle=\sffamily\bfseries, %\texttt, %
%captionpos=b,%
tabsize=3,%
frame=lines,%
numbers=left,%
numberstyle=\tiny,%
numbersep=5pt,%
breaklines=true,%
%morekeywords={descriptor_gaussian,descriptor,partition,fcm_possibilistic,dataset,my_exception,exception,std,vector},%
escapeinside={@*}{*@},% 
}

% % % % % % % % % % % % % % % % % % % % % % % % % % % % % % % % % % % 
% package minted
%\usepackage{minted}

% This package requires a special command line option in compilation
% pdflatex -shell-escape thesis

%%%%%%%%%%%%%%%%%%%%%%%%%%%%%%%%%%%%%%%%%%%%%%%%%%%%%%%%%%%%%%%%%%%%%


%%%% TODO LIST GENERATOR %%%%%%%%%

\usepackage{color}
\definecolor{brickred}      {cmyk}{0   , 0.89, 0.94, 0.28}

\makeatletter \newcommand \kslistofremarks{\section*{Remarks} \@starttoc{rks}}
  \newcommand\l@uwagas[2]
    {\par\noindent \textbf{#2:} %\parbox{10cm}
{#1}\par} \makeatother


\newcommand{\ksremark}[1]{%
{%\marginpar{\textdbend}
{\color{brickred}{[#1]}}}%
\addcontentsline{rks}{uwagas}{\protect{#1}}%
}










%%%%%%%%%%%%%% END OF TODO LIST GENERATOR %%%%%%%%%%%  

%%%%%%%%%%%% FANCY HEADERS %%%%%%%%%%%%%%%
% do not capitalise fancy headers
\usepackage{fancyhdr}
\pagestyle{fancy}
\fancyhf{}
\fancyhead[LO]{\nouppercase{\it\rightmark}}
\fancyhead[RE]{\nouppercase{\it\leftmark}}
\fancyhead[LE,RO]{\it\thepage}


\fancypagestyle{onlyPageNumbers}{%
   \fancyhf{} 
   \fancyhead[LE,RO]{\it\thepage}
}

\fancypagestyle{noNumbers}{%
   \fancyhf{} 
   \fancyhead[LE,RO]{}
}


\fancypagestyle{PageNumbersChapterTitles}{%
   \fancyhf{} 
   \fancyhead[LO]{\nouppercase{\Firstnames \Surname}}
   \fancyhead[RE]{\nouppercase{\leftmark}} 
   \fancyfoot[CE, CO]{\thepage}
}
 



%%%%%%%%%%%%%%%%%%%%%%%%%%%







\newcounter{pagesWithoutNumbers}

%%%%%%%%%%%%%%%%%%%%%%%%%%% 
\usepackage{xstring}
\usepackage{ifthen}
\newcommand{\printOpiekun}[1]{%		

    \StrLen{\Consultant}[\mystringlen]
    \ifthenelse{\mystringlen > 0}%
    {%
       {\large{\bfseries CONSULTANT}\par}
       
       {\large{\bfseries \Consultant}\par}
    }%
    {}
} 
%
%%%%%%%%%%%%%%%%%%%%%%%%%%%%%%%%%%%%%%%%%%%%%%
 
% Please do not modify the lines below!
\author{\Firstnames\ \Surname}
\newcommand{\Author}{\Firstnames\ \MakeUppercase{\Surname}}
\newcommand{\Type}{MASTER THESIS}
\newcommand{\Faculty}{Faculty of Automatic Control, Electronics and Computer Science}
\newcommand{\Polsl}{Silesian University of Technology}
\newcommand{\Logo}{politechnika_sl_logo_bw_pion_en.pdf}
\newcommand{\LeftId}{Student identification number}
\newcommand{\LeftProgram}{Programme}
\newcommand{\LeftSpecialisation}{Specialisation}
\newcommand{\LeftSUPERVISOR}{SUPERVISOR}
\newcommand{\LeftDEPARTMENT}{DEPARTMENT}
%%%%%%%%%%%%%%%%%%%%%%%%%%%%%%%%%%%%%%%%%%%%%%


%%%%%%%%%%%%%%%%%%%%%%%%%%%%%%%%%%%%%%%%%%%%%%%
%                                             %
% END OF SETTINGS                             %
%                                             %
%%%%%%%%%%%%%%%%%%%%%%%%%%%%%%%%%%%%%%%%%%%%%%%

% Add keywords for code snippets if you need
\lstset{%
morekeywords={string,exception,std,vector},% keyword for the listings package
}


%%%%%%%%%%%%%%%%%%%%%%%%%%%%%%%%%%%%%%%%%%%%%%%
%                                             %
% CUSTOMISE YOUR THESIS                       %
%                                             %
%%%%%%%%%%%%%%%%%%%%%%%%%%%%%%%%%%%%%%%%%%%%%%%
% TODO
\newcommand{\Firstnames}{First Names}
\newcommand{\Surname}{Surname}
\newcommand{\Supervisor}{$\langle$title first name surname$\rangle$}     % remove \langle and \rangle in the final version
\newcommand{\Title}{Thesis title in English}
\newcommand{\TitleAlt}{Thesis title in Polish}
\newcommand{\Program}{Control, Electronic, and Information Engineering}
\newcommand{\Specialisation}{Informatics}
\newcommand{\Id}{$\langle$your student id$\rangle$}                      % remove \langle and \rangle in the final version
\newcommand{\Departament}{$\langle$departament name$\rangle$}            % remove \langle and \rangle in the final version

% If you have a consultant for your thesis, put their name below ...
\newcommand{\Consultant}{$\langle$title first name surname$\rangle$}     % remove \langle and \rangle in the final version
% ... else leave the braces empty:
%\newcommand{\consultant}{} % no consultant

%%%%%%%%%%%%%%%%%%%%%%%%%%%%%%%%%%%%%%%%%%%%%%%
%                                             %
% END OF CUSTOMISATION                        %
%                                             %
%%%%%%%%%%%%%%%%%%%%%%%%%%%%%%%%%%%%%%%%%%%%%%%

\begin{document}
%\kslistofremarks 

%%%%%%%%%%%%%%%%%%%%%%%%%%%%%%%%%%%%%%%%%%%%%%%
%                                             %
% TITLE PAGE                                  %
% DO NOT MODIFY!                              %
%                                             %
%%%%%%%%%%%%%%%%%%%%%%%%%%%%%%%%%%%%%%%%%%%%%%%

%%%%%%%%%%%%%%%%%%  TITLE PAGE %%%%%%%%%%%%%%%%%%%
\pagestyle{empty}
{
	\newgeometry{top=1.5cm,%
	             bottom=2.5cm,%
	             left=3cm,
	             right=2.5cm}
 
	\ifxetex 
	  \begingroup
	  \setsansfont{Calibri}
	   
	\fi 
	 \sffamily
	\begin{center}
	\includegraphics[width=50mm]{\Logo}
	 
	
	{\Large\bfseries\Type\par}
	
	\vfill  \vfill  
			 
	{\large\Title\par}
	
	\vfill  
		
	{\large\bfseries\Author\par}
	
	{\normalsize\bfseries \LeftId: \Id}
	
	\vfill  		
 
	{\large{\bfseries \LeftProgram:} \Program\par} 
	
	{\large{\bfseries \LeftSpecialisation:} \Specialisation\par} 
	 		
	\vfill  \vfill 	\vfill 	\vfill 	\vfill 	\vfill 	\vfill  
	 
	{\large{\bfseries \LeftSUPERVISOR}\par}
	
	{\large{\bfseries \Supervisor}\par}
				
	{\large{\bfseries \LeftDEPARTMENT\ \Departament} \par}
		
	{\large{\bfseries \Faculty}\par}
		
	\vfill  \vfill  

    	
    \printOpiekun{\Consultant}
    
	\vfill  \vfill  
		
    {\large\bfseries  Gliwice \the\year}

   \end{center}	
       \ifxetex 
       	  \endgroup
       \fi
	\restoregeometry
}
  
%%%%%%%%%%%%%%%%%%%%%%%%%%%%%%%%%%%%%%%%%%%%%%%
%                                             %
% END OF TITLE PAGE                           %
%                                             %
%%%%%%%%%%%%%%%%%%%%%%%%%%%%%%%%%%%%%%%%%%%%%%%

\cleardoublepage
 
\rmfamily\normalfont
\pagestyle{empty}

  
% So we start with the thesis :-)
% TODO

\subsubsection*{Thesis title}  
\Title

\subsubsection*{Abstract} 
(Thesis abstract – to be copied into an appropriate field during an electronic submission – in English.)

\subsubsection*{Key words}  
(2-5 keywords, separated by commas)

\subsubsection*{Tytuł pracy}
\begin{otherlanguage}{polish}
\TitleAlt
\end{otherlanguage}

\subsubsection*{Streszczenie} 
\begin{otherlanguage}{polish}
(Streszczenie pracy – odpowiednie pole w systemie APD powinno zawierać kopię tego streszczenia.)
\end{otherlanguage}

\subsubsection*{Słowa kluczowe} 
\begin{otherlanguage}{polish}
(2-5 slow (fraz) kluczowych, oddzielonych przecinkami)
\end{otherlanguage}


%%%%%%%%%%%%%%%%%% Table of contents %%%%%%%%%%%%%%%%%%%%%%
%\pagenumbering{Roman}
\thispagestyle{empty}
\tableofcontents
\thispagestyle{empty}

%%%%%%%%%%%%%%%%%%%%%%%%%%%%%%%%%%%%%%%%%%%%%%%%%%%%%
\setcounter{pagesWithoutNumbers}{\value{page}}
\mainmatter
\pagestyle{empty}
 
\cleardoublepage

\pagestyle{PageNumbersChapterTitles}

%%%%%%%%%%%%%% body of the thesis %%%%%%%%%%%%%%%%%

% TODO
\chapter{Introduction}

%\begin{itemize}
%\item introduction into the problem domain
%\item settling of the problem in the domain
%\item objective of the thesis 
%\item scope of the thesis
%\item short description of chapters
%\item clear description of contribution of the thesis's author
%\end{itemize}

% TODO
\chapter{[Chapter title]}


%\begin{itemize}
%\item What problem do I want (have to :-) to solve?
%\item Why the problem is important?
%\item How do others solve the problem?
%\item What are pros and cons of my solution?
%\end{itemize}

References to 
book \cite{bib:book},
scientific papers in journals \cite{bib:article},
conference papers \cite{bib:conference},
and web pages \cite{bib:internet}.

Equations should be numbered:
\begin{align}
y = \frac{\partial x}{\partial t}
\end{align}

%\chapter{[Problem analysis]}
%
%\begin{itemize}
%\item problem analysis, problem statement
%\item state of the art, literature research (all sources in the thesis have to be referenced, eg journal article \cite{bib:article} book \cite{bib:book}, conference paper \cite{bib:conference}, internet source \cite{bib:internet})
%\item description of known solutions, algorithms
%\item location of the thesis in scientific domain
%\item The title of this chapter is similar to the title of the thesis.
%\end{itemize}

% TODO
\chapter{[Chapter title]}

text

\section{[Section title]}

\section{[Subsection title]}

Each figure in the document should be referred to at least once (fig. \ref{fig:2}).

\begin{figure}
\centering
\begin{tikzpicture}
\begin{axis}[
    y tick label style={
        /pgf/number format/.cd,
            fixed,   % po zakomentowaniu os rzednych jest indeksowana wykladniczo
            fixed zerofill, % 1.0 zamiast 1
            precision=1,
        /tikz/.cd
    },
    x tick label style={
        /pgf/number format/.cd,
            fixed,
            fixed zerofill,
            precision=2,
        /tikz/.cd
    }
]
\addplot [domain=0.0:0.1] {rnd};
\end{axis} 
\end{tikzpicture}
\caption{Figure caption.} % Figure caption is BELOW the figure!
\label{fig:2}
\end{figure}

Each table in the document should be referred to at least once (Tab. \ref{tab:results}).

\begin{table}
\centering
\caption{A caption of a table is ABOVE it.}
\label{tab:results}
\begin{tabular}{rrrrrrrr}
\toprule
	         &                                     \multicolumn{7}{c}{method}                                      \\
	         \cmidrule{2-8}
	         &         &         &        \multicolumn{3}{c}{alg. 3}        & \multicolumn{2}{c}{alg. 4, $\gamma = 2$} \\
	         \cmidrule(r){4-6}\cmidrule(r){7-8}
	$\zeta$ &     alg. 1 &   alg. 2 & $\alpha= 1.5$ & $\alpha= 2$ & $\alpha= 3$ &   $\beta = 0.1$  &   $\beta = -0.1$ \\
\midrule
	       0 &  8.3250 & 1.45305 &       7.5791 &    14.8517 &    20.0028 & 1.16396 &                       1.1365 \\
	       5 &  0.6111 & 2.27126 &       6.9952 &    13.8560 &    18.6064 & 1.18659 &                       1.1630 \\
	      10 & 11.6126 & 2.69218 &       6.2520 &    12.5202 &    16.8278 & 1.23180 &                       1.2045 \\
	      15 &  0.5665 & 2.95046 &       5.7753 &    11.4588 &    15.4837 & 1.25131 &                       1.2614 \\
	      20 & 15.8728 & 3.07225 &       5.3071 &    10.3935 &    13.8738 & 1.25307 &                       1.2217 \\
	      25 &  0.9791 & 3.19034 &       5.4575 &     9.9533 &    13.0721 & 1.27104 &                       1.2640 \\
	      30 &  2.0228 & 3.27474 &       5.7461 &     9.7164 &    12.2637 & 1.33404 &                       1.3209 \\
	      35 & 13.4210 & 3.36086 &       6.6735 &    10.0442 &    12.0270 & 1.35385 &                       1.3059 \\
	      40 & 13.2226 & 3.36420 &       7.7248 &    10.4495 &    12.0379 & 1.34919 &                       1.2768 \\
	      45 & 12.8445 & 3.47436 &       8.5539 &    10.8552 &    12.2773 & 1.42303 &                       1.4362 \\
	      50 & 12.9245 & 3.58228 &       9.2702 &    11.2183 &    12.3990 & 1.40922 &                       1.3724 \\
\bottomrule
\end{tabular}
\end{table}  


%\chapter{[Subject of the thesis]}
%
%\begin{itemize}
%\item solution to the problem proposed by the author of the thesis
%\item theoretical analysis of proposed solutions
%\item rationale of applied methods, algorithms, and tools
%\end{itemize}



% TODO
%\chapter{Experiments}
%
%This chapter presents the experiments. It is a crucial part of the thesis and has to dominate in the thesis. 
%The experiments and their analysis should be done in the way commonly accepted in the scientific community (eg. benchmark datasets, cross validation of elaborated results, reproducibility and replicability of tests etc).
%
%
%\section{Methodology}
%
%\begin{itemize}
%\item description of methodology of experiments
%\item description of experimental framework (description of user interface of research applications – move to an appendix)
%\end{itemize}
%
%
%\section{Data sets}
%
%\begin{itemize}
%\item description of data sets
%\end{itemize}
%
%
%\section{Results}
%
%\begin{itemize}
%\item presentation of results, analysis and wide discussion of elaborated results, conclusions
%\end{itemize}
%
%
%
%\begin{table}
%\centering
%\caption{A caption of a table is ABOVE it.}
%\label{id:tab:wyniki}
%\begin{tabular}{rrrrrrrr}
%\toprule
%	         &                                     \multicolumn{7}{c}{method}                                      \\
%	         \cmidrule{2-8}
%	         &         &         &        \multicolumn{3}{c}{alg. 3}        & \multicolumn{2}{c}{alg. 4, $\gamma = 2$} \\
%	         \cmidrule(r){4-6}\cmidrule(r){7-8}
%	$\zeta$ &     alg. 1 &   alg. 2 & $\alpha= 1.5$ & $\alpha= 2$ & $\alpha= 3$ &   $\beta = 0.1$  &   $\beta = -0.1$ \\
%\midrule
%	       0 &  8.3250 & 1.45305 &       7.5791 &    14.8517 &    20.0028 & 1.16396 &                       1.1365 \\
%	       5 &  0.6111 & 2.27126 &       6.9952 &    13.8560 &    18.6064 & 1.18659 &                       1.1630 \\
%	      10 & 11.6126 & 2.69218 &       6.2520 &    12.5202 &    16.8278 & 1.23180 &                       1.2045 \\
%	      15 &  0.5665 & 2.95046 &       5.7753 &    11.4588 &    15.4837 & 1.25131 &                       1.2614 \\
%	      20 & 15.8728 & 3.07225 &       5.3071 &    10.3935 &    13.8738 & 1.25307 &                       1.2217 \\
%	      25 &  0.9791 & 3.19034 &       5.4575 &     9.9533 &    13.0721 & 1.27104 &                       1.2640 \\
%	      30 &  2.0228 & 3.27474 &       5.7461 &     9.7164 &    12.2637 & 1.33404 &                       1.3209 \\
%	      35 & 13.4210 & 3.36086 &       6.6735 &    10.0442 &    12.0270 & 1.35385 &                       1.3059 \\
%	      40 & 13.2226 & 3.36420 &       7.7248 &    10.4495 &    12.0379 & 1.34919 &                       1.2768 \\
%	      45 & 12.8445 & 3.47436 &       8.5539 &    10.8552 &    12.2773 & 1.42303 &                       1.4362 \\
%	      50 & 12.9245 & 3.58228 &       9.2702 &    11.2183 &    12.3990 & 1.40922 &                       1.3724 \\
%\bottomrule
%\end{tabular}
%\end{table}  
%
%\begin{figure}
%\centering
%\begin{tikzpicture}
%\begin{axis}[
%    y tick label style={
%        /pgf/number format/.cd,
%            fixed,   % po zakomentowaniu os rzednych jest indeksowana wykladniczo
%            fixed zerofill, % 1.0 zamiast 1
%            precision=1,
%        /tikz/.cd
%    },
%    x tick label style={
%        /pgf/number format/.cd,
%            fixed,
%            fixed zerofill,
%            precision=2,
%        /tikz/.cd
%    }
%]
%\addplot [domain=0.0:0.1] {rnd};
%\end{axis} 
%\end{tikzpicture}
%\caption{Figure caption is BELOW the figure.}
%\label{fig:2}
%\end{figure}
%
%\begin{figure}
%\begin{lstlisting}
%if (_nClusters < 1)
%	throw std::string ("unknown number of clusters");
%if (_nIterations < 1 and _epsilon < 0)
%	throw std::string ("You should set a maximal number of iteration or minimal difference -- epsilon.");
%if (_nIterations > 0 and _epsilon > 0)
%	throw std::string ("Both number of iterations and minimal epsilon set -- you should set either number of iterations or minimal epsilon.");
%\end{lstlisting}
%\caption{Example of pseudocode.}
%\end{figure}

% TODO
\chapter{Summary}

%\begin{itemize}
%\item What problem have I solved?
%\item How have I solved the problem?
%\item What are pros and cons of my solutions?
%\item Can I state some recommendations?
%\end{itemize}

\begin{itemize}
\item synthetic description of performed work
\item conclusions
\item  future development, potential future research
\item Has the objective been reached?
\end{itemize}

\backmatter 

%\bibliographystyle{plain} % bibtex
%\bibliography{biblio} % bibtex
\printbibliography           % biblatex 
\addcontentsline{toc}{chapter}{References}

\begin{appendices}

% TODO
\chapter{Technical documentation}

% TODO
\chapter{List of abbreviations and symbols}

\begin{itemize}
\item[DNA] deoxyribonucleic acid
\item[MVC] model--view--controller 
\item[$N$] cardinality of data set
\item[$\mu$] membership function of a fuzzy set
\item[$\mathbb{E}$] set of edges of a graph
\item[$\mathcal{L}$] Laplace transformation
\end{itemize}

% TODO
% List of additional files. If you do not have any additional file, just comment this chapter.
\chapter{List of additional files in~electronic submission (if applicable)}

Additional files uploaded to the system include:
\begin{itemize}
\item source code of the application,
\item test data,
\item a video file showing how software or hardware developed for thesis is used,
\item etc.
\end{itemize}

\listoffigures
\addcontentsline{toc}{chapter}{List of figures}
\listoftables
\addcontentsline{toc}{chapter}{List of tables}
	
\end{appendices}

\end{document}


%% Finis coronat opus.

